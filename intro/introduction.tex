\chapter{序論}
冷却原子の運動状態と内部状態をレーザーで制御することで量子コンピュータや量子シミュレーションが実現できることが提案されて久しい\cite{Cirac_1995}.冷却原子を準備する装置の一つに静電場と高周波電場による電磁場を利用してイオンを空間中に閉じ込めるPaulトラップと呼ばれる装置が挙げられる.Paulトラップは量子情報処理の実験の多くに使用されている.現在,一列に並ぶイオンを用いた79-qubitの安定な制御が可能となっている\cite{Wright_2019}.また,イオンの配列を一列から二次元へと拡張することで量子コンピュータや量子シミュレーションの応用範囲が広がることが指摘されており\cite{Cirac_2000},二次元配列として捕獲されたイオンを用いてフラストレーション効果を調査した報告などがある\cite{Mielenz_2016}.

電極の非対称性などに起因してトラップ内に発現する浮遊電場の存在はイオンに余剰マイクロ運動と呼ばれる微小振動を引き起こす.安定な量子系を確立するにあたり,余剰マイクロ運動がよく補正されていることは前提条件として要求される.現在,一列配列イオンの余剰マイクロ運動の補正方法は多数報告されているが\cite{Berkeland_1998}\cite{Chen_2020}\cite{Timm_2015},二次元配列のイオンの余剰マイクロ運動の補正手法は確立されていない.

イオンは電場の影響を敏感に受ける.そのため,単一イオンをセンサーとしてトラップ内の浮遊電場の解析\cite{Narayanan_2011}や電極表面に存在する電気ノイズの解析\cite{Danii_2014}が行われている.様々な構造を持つ電極によってイオンの制御が試みられている中,各トラップに存在するノイズの検出は非常に重要となっている.

当研究グループでは,微細加工技術を施したプレーナートラップを用いてイオンを二列に捕獲することを可能としており\cite{Tanaka_2021},二列配列イオンにおける電場の検出を試みている.よく制御された二列配列イオンが確立されれば,量子コンピュータや量子シミュレーションの大規模化に加え,イオン列間に生じる電位障壁に対するトンネル効果の実証実験や,巨視的現象である摩擦の原子レベルでの検証実験などへの応用が期待される\cite{Tanaka_2021}\cite{L-Timm_2021}.