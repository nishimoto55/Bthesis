\chapter{概要}
二列配列イオンの余剰マイクロ運動の抑制のためには,浮遊電場を補正dc電圧によって補正しなければいけない.本研究では,浮遊電場の検出に向けてプレーナートラップ上の電場を評価する2つの手法の開発を行った.1つ目は,複数個イオンを用いた手法である.トラップされたイオンは電場から受ける力とクーロン相互作用によって受ける力とがつり合う位置が平衡位置となる.イオン捕獲画像からイオンの位置を抽出することでイオン捕獲位置における電場が求められる.そしてイオンの個数が異なる場合の電場を繰り返し計算し,フィッティングを行うことで電場の傾きの算出を行った.2つ目は,単一イオンを用いる手法である.単一イオンに発振信号を与えたときに観測される共鳴現象によるイオンの振幅の広がりから永年周波数を決定し,電場の傾きの算出を行った.一列配列イオンにおいて上記の2つの手法を用いて求められた電場の傾きが15 $\sim$ 16 \%の精度で一致することが確かめられた.そして,実験から得られた2つの電場の傾きとMathematicaによるシミュレーションで得られた電場の傾きと間に差が現れた.また,二列配列イオンにおいてイオン列間距離の実験値とシミュレーション値の比較と共鳴現象の確認を行った.