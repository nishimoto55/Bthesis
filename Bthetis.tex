\documentclass[a4j,10.5pt,titlepage]{jarticle}
\usepackage[utf8]{inputenc}

\usepackage[dvipdfmx]{graphicx}
\usepackage{wrapfig}
\usepackage{amsmath}
\usepackage{geometry}
\usepackage{comment}
\usepackage{bm}
\usepackage{fancyvrb}

\usepackage[dvipdfmx]{hyperref}
% for hyperref
\usepackage{pxjahyper}
\hypersetup{% hyperrefオプションリスト
setpagesize=false,
 bookmarksnumbered=true,%
 bookmarksopen=true,%
 colorlinks=true,%
 linkcolor=blue,
 citecolor=blue,
}

% ページの余白を1.25インチにする
\geometry{
	left=1.25truein,
	right=1.25truein,
	top=1.25truein,
	bottom=1.25truein,
}

%ページの上下に出力される図と図の間のスペース
\setlength\floatsep{5.0pt} %dblfloatsep

%ページの上下に出力される図と本文の間のスペース
\setlength\textfloatsep{5.0pt} %dbltextfloatsep

%ページの途中に出力される図と本文の間のスペース
\setlength\intextsep{5.0pt}

%図の参照
\newcommand{\Fig}[1]{Fig.\ref{fig:#1}}
%表の参照
\newcommand{\Tb}[1]{Tab.\ref{tab:#1}}
%式の参照
\newcommand{\Eq}[1]{Eq.(\ref{eq:#1})}

\renewcommand{\figurename}{Figure}
\renewcommand{\tablename}{Table}

\makeatletter
 \renewcommand{\theequation}{%
   \thechapter.\arabic{equation}}
  \@addtoreset{equation}{chapter}
  
  \renewcommand{\thefigure}{
  \thechapter.\arabic{figure}}
  \@addtoreset{figure}{chapter}
  
  \renewcommand{\thetable}{
    \thechapter.\arabic{table}}
  \@addtoreset{table}{chapter}
\makeatother

\title{卒業論文}
\author{Ryohsuke Nishimoto}
\date{\today}

\begin{document}
\maketitle
\pagenumbering{roman}
\setcounter{tocdepth}{3}
\tableofcontents

\clearpage
\pagenumbering{arabic}

\section{序論}
	%\input{intro/intro}
\clearpage
\section{理論}
	\subsection{イオントラップ}
	\subsection{パウルトラップ}
	\subsection{プレーナーイオントラップ}
		\subsubsection{電極の仕様}
		\subsubsection{Single-well}
		\subsubsection{Double-well}
	\subsection{レーザー冷却}
	\subsection{イオンの運動}
		\subsubsection{余剰マイクロ運動}
	\subsection{画像処理によるイオン捕獲位置と電場の算出}
		\subsubsection{グレースケール化,二値化}
		\subsubsection{イオンの検出および,捕獲位置と電場の算出方法}
\clearpage
\section{実験のセットアップ}
実験系とレーザーの情報
\clearpage
\section{実験方法と結果}
	\subsection{一列配列イオン}
		\subsubsection{捕獲の手順}
		\subsubsection{イオン捕獲位置のdc電圧依存性}
		\subsubsection{永年周波数のdc電圧依存性}
		\subsubsection{シミュレーションとの比較}
	\subsection{二列配列イオン}
		\subsubsection{捕獲の手順}
		\subsubsection{比率Rとイオン列間距離の関係}
		\subsubsection{シミュレーションとの比較}
	
	\subsection{余剰マイクロ運動の補正}
		\subsubsection{補正の手順}
		\subsubsection{補正結果}
\clearpage
\section{考察}
\clearpage
\section{結論と展望}
\clearpage
\section*{謝辞}
\clearpage
\begin{thebibliography}{99}
\end{thebibliography}


%\input{section2/text_section2}
%\input{section3/text_section3}
%\input{section4/text_section4}
%\input{section5/text_section5}
%\input{section7/text_section7}
%\input{BeamProfile/BeamProfile}
%\input{section6/text_section6}

\end{document}


%%%%%%%%%%%%%%%%%%%%%%%%%%%%%%%%%%%%%%%%%%%%%%%%%%%%%%%%%%%%%%%%%%%%%%%%%%%%%%%%%%%%%%%%

