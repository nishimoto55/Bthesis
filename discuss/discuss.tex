\chapter{考察}
\section{一列配列イオン}
本報告では複数個イオンのそれぞれの位置での電場から求めた電場の傾きと,単一イオンの永年周波数から求めた電場の傾きの比較を3種類のdc電圧セットに対して行った.異なるdc電圧セットに対しても同様に電場の評価を行うことでプレーナートラップ上に発生する電場のマッピング可能であると考える.また,実験から得られた電場の傾きとシミュレーションから得られる電場の傾きの間に大きな差が存在することが確認され,その要因としては
\begin{itemize}
	\item 電圧降下によって生じる,印加する電圧とプレーナートラップに印加される電圧との差異
	\item ac成分を持つdc電圧
	\item 二列配列イオンの捕獲を可能とするために配置しているcenter-rf電極に印加されてしまうrf電圧の存在
	\item 電極作製時における機械的な不完全性や電極の非対称性ならびに電極表面に付着する不純物によるパッチポテンシャルなどに起因する浮遊電場の存在
\end{itemize}
などが考えられる.実験値とシミュレーション値との差を浮遊電場とみなして補正dc電圧によってこれを補正する予定であったが,上述のように浮遊電場以外の要因も考えられることから,実験的に確かめる必要があると考えている.また,本実験ではz方向に関して電場の検出を行ってきたため,x,y方向への拡張も必要であると考えている.
	
シミュレーション方法について,印加するdc電圧を変化させたときのイオンの変位が一致することは確認できたが,実験とシミュレーションにおけるプレーナートラップの座標の絶対値の一致は確認できなかった.これはイオンを高倍率で検出しているが故にCCDカメラで集光する範囲がに対応するプレーナートラップの範囲の決定ができなかったためである.また,y方向からイオンの蛍光を集光するためにイオンの表面電極からの距離の詳細な情報の抽出ができない.したがって,実験とシミュレーションにおけるプレーナートラップの座標の絶対値を一致させることでシミュレーションの精度が向上すると考えられる.

\section{二列配列イオン}
二列配列イオンではイオンがななめに捕獲されることが確認でき,二列配列イオンの各イオン列の中心は一致しなかった.共鳴現象の観測にあたり上の列と下の列においてシミュレーション上でその共鳴周波数が一致していることに対して,実験結果では20 kHz程度の差が確認できた.このことから,永年周波数の実験値とシミュレーション値との比較などにより,各イオン列の中心が一致しないことの原因の特定が必要であると考えている.また,集光系のピントを変化させたときに上の列と下の列でイオンの蛍光強度が強くなるピントの位置が異なることが確認できた.これより,各イオン列の電極表面からの距離が異なっていると考えられる.