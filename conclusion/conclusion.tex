\chapter{結論と展望}
安定な量子系の拡張に向け,まず二列配列における余剰マイクロ運動の抑制に向けプレーナートラップに存在する浮遊電場の検出を行う必要があり,これを実験とシミュレーションとの差で決定し補正dc電圧で補正する.そのために実験系における電場の検出方法の開発を行った.電場検出方法の開発にあたり複数個イオンを用いる手法と単一イオンを用いる手法の2つの手法によるクロスチェックを行いその妥当性を確かめた.その結果一列配列イオンのz方向に関して,2つの手法から求められる電場の傾きは15 $\sim$ 16 \%の精度で一致することが確認され,単一イオンを用いてプレーナートラップ上の電場の解析を行うことができることが確認された.一列配列イオンにおいてx方向, y方向についても同様に電場の検出を行うことができる.そして,単一イオンを用いた電場の検出を二列配列イオンにも適用することを考えており,現在z方向における共鳴現象が観測されている.