\chapter{結論と展望}
安定な量子系の拡張に向け,二列配列イオンの余剰マイクロ運動の抑制を目標に研究を行った.そのためにはプレーナートラップ上の電場を評価しシミュレーション値と比較することで浮遊電場の検出を行い,補正dc電圧によって打ち消すことが必要となる.そこでプレーナートラップ上の電場を評価する2通りの手法の開発を行った.1つ目は複数個イオンを用いた手法(\ref{alpha_pos}節)であり,2つ目は単一イオンを用いた手法(\ref{alpha_SecFreq}節)である.\ref{COMP_alpha}節で示したように2通りの手法で求められる電場の傾き$\alpha$が一致した.一列配列イオンに対して二列配列イオンの捕獲は実験的に難しいことから,単一イオンを用いた手法によって二列配列イオンにおける電場の傾きを求めることでプレーナートラップ上の電場の評価が可能であると考えられる.また現在,二列配列イオンにおいて共鳴現象の確認ができている(\ref{2D_res}節).

浮遊電場の検出のため,電場の傾きについて実験値とシミュレーション値との比較を行い,シミュレーションの妥当性を確認した.比較に用いた量は,
\begin{enumerate}
\item イオン捕獲位置の変位
\item z方向の永年周波数
\item $R$ - $d$特性
\end{enumerate}
である.一列配列イオンを捕獲する条件にて単一イオンを用いることで(1),(2)の実験値を得た.そして,二列配列イオンを用いることで(3)の測定を行った.
\begin{description}
\item [イオン捕獲位置の変位]
一列配列イオンを用いて実験を行った(\ref{quantity_1}節).プレーナートラップに印加するdc電圧を変化させたときのイオン捕獲位置の変位を実験値とシミュレーション値の比較を行った.このとき,シミュレーションでは形成されるポテンシャルの極小値をイオンの捕獲位置として比較した.
\item[$z$方向の永年周波数]
一列配列イオンを用いて実験を行った(\ref{quantity_2}節).dc電圧の変化に対するイオン捕獲位置の変位を比較した結果では一致することが確認できたが,z方向の永年周波数の比較を行うと20 $\sim$ 40 kHz程度の差が現れることが確認された.また,rf擬ポテンシャルの影響は無視できることが確認できた.
\item[$R$ - $d$特性]
二列配列イオンを用いて実験を行った(\ref{quantity_3}節).実験値とシミュレーション値との差が現れた.また,二列配列イオンの上の列と下の列で共鳴現象が確認できる周波数が異なった.これらはrf電圧とcenter-rf電圧のカップリングが行われることで$R$の値が不安定になっていることが原因の一つとして考えられる.
\end{description}
以上より,実験とシミュレーションとの間に差異が生じていることが分かる.特に$z$方向の永年周波数の実験値とシミュレーション値との差は,プレーナートラップ上に形成されているポテンシャルの概形を反映していることから,浮遊電場の検出に際して非常に重要な情報となる.この差異には,浮遊電場以外に電圧降下による印加電圧の低下や電極表面に付着した不純物によるパッチポテンシャルなどが含まれているため,実験的に補正dc電圧を印加することで浮遊電場であるかどうかを確かめる必要がある.また,実験とシミュレーションとの差を誘発するその他の原因の究明あるいはその他のシミュレーション方法の導入などを行うことで精度の向上が期待できる.

実験とシミュレーションとの差異の原因の一つに,プレーナートラップの電極形状によるものが存在する.これはプレーナートラップのサイズが小さくなるほど影響が強くなる.そこで,測定されたz方向の永年周波数を用いてend1, end3およびend2, end4をそれぞれ一組として永年周波数の変化率を確認した.このとき,$f \simeq 1.1$で変化率が一致した.end2, end4に比べてend1, end3の電極が小さいことから非対称に電圧を印加した場合に永年周波数の変化率が一致していると考える.また,電圧値が非対称となる理由に,その他の電極に存在するオフセット電圧の存在も挙げられる.したがって,電極に任意のdc電圧を印加しよく制御するために使用しているプレーナートラップの挙動を理解する必要があると結論づけられる.

単一イオンを用いて測定できる永年周波数から電場の傾きが得られることから,二列配列イオンの条件で電場の評価が可能である.これより,一列配列イオンおよび二列配列イオンの条件で電場の評価を行うことで広い範囲でプレーナートラップ上の電場のマッピングを行うことができる.また,二列配列イオンではイオン列間距離がある値を閾値として振動モードの縮退が解けることが理論的に示されており\cite{Welzel_2011},本研究で開発した手法を用いることで二列配列イオンにおける振動モードの実測が可能であると考えられる.